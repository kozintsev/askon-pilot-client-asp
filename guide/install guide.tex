\documentclass[12pt,a4paper,titlepage]{article}
\usepackage[utf8]{inputenc}
\usepackage[russian]{babel}
\usepackage[OT1]{fontenc}
\usepackage{amsmath}
\usepackage{amsfonts}
\usepackage{amssymb}
\usepackage{listings}
\usepackage{graphicx}
%%\author{Богданов Н.Е.}
\title{Руководство по установке}
\begin{document}
\maketitle
\tableofcontents
\newpage
\section{Установка под Windows}
\subsection{Установка модулей служб IIS и ASP.NET}
\subsubsection{Установка .NET 3.5 и .NET 4.5 в Windows Server 2012, Windows 8 или Windows 7}
\begin{enumerate}
\item На начальном экране правой кнопкой мыши щелкните плитку Командная строка, а затем щелкните \textbf{Запуск от имени администратора.}
\item В командной строке введите следующую команду: \textbf{dism /online /enable-feature /featurename:netfx3} для установки .NET 4.5 замените netfx3 на netfx45.
\item Дождитесь завершения выполнения команды. Это может занять несколько минут.
\item Закройте окно командной строки.
\end{enumerate}
Дополнительные сведения об установке и новые версии .NET Framework можно посмотреть по ссылке:
\textbf{\\https://msdn.microsoft.com/ru-ru/library/5a4x27ek(v=vs.110).aspx}
\subsubsection{Установка служб IIS на Windows Server® 2012 с помощью пользовательского интерфейса диспетчера служб IIS.}
\begin{enumerate}
\item На начальном экране щелкните плитку Диспетчер сервера, а затем нажмите кнопку ОК.
\item В диспетчере сервера выберите Панель мониторинга и щелкните Добавить роли и компоненты.
\item В мастере добавления ролей и компонентов на странице Перед началом работы нажмите кнопку Далее.
\item На странице Выбор типа установки выберите "Установка ролей или компонентов" и нажмите кнопку Далее.
\item На странице Выбор целевого сервера выберите Выбор сервера из пула серверов, выберите имя своего сервера и нажмите кнопку Далее.
\item На странице Выбор ролей сервера укажите Веб-сервер (IIS) и нажмите кнопку Далее.
\item На странице Выбор компонентов нажмите кнопку Далее.
\item На странице Роль веб-сервера (IIS) нажмите кнопку Далее.
\item На странице Выбор служб ролей просмотрите выбранные по умолчанию службы ролей, разверните узел Разработка приложений и выберите ASP.NET 4.5 (если вы установили .NET 3.5, также выберите ASP.NET 3.5).
\item На странице Сводка компонентов для установки подтвердите свой выбор, а затем нажмите кнопку Установить.
\item В области Добавить компоненты, которые требуются для ASP.NET 4.5? нажмите кнопку Добавить компоненты.\\Будут добавлены следующие дополнительные компоненты:
	\begin{itemize}
		\item .NET Extensibility 4.5.
		\item Расширения ISAPI.
		\item Фильтры ISAPI.
		\item .NET Extensibility 3.5 (если было выбрано ASP.NET 3.5).
	\end{itemize}
\item Нажмите кнопку Далее.
\item На странице Подтверждение выбранных элементов для установки нажмите кнопку Установить.
\item На странице Ход выполнения установки убедитесь, что установка роли веб-сервера (IIS) и требуемых служб ролей успешно завершена, а затем нажмите кнопку Закрыть.
Чтобы убедиться, что службы IIS успешно установлены, введите в веб-браузер следующее:
\\\textbf{http://localhost}\\
Откроется страница приветствия IIS по умолчанию.
\end{enumerate}
\subsubsection{Установка служб IIS и модулей ASP.NET на Windows 7, 8  с помощью пользовательского интерфейса}
\begin{enumerate}
\item На начальной щелкните Панель управления.
\item В панели управления выберите Программы (в Windows 7, Программы и компоненты), а затем Включение и отключение компонентов Windows.
\item Чтобы установить компоненты по умолчанию, в диалоговом окне Компоненты Windows выберите Службы IIS.
Чтобы добавить компоненты, которые поддерживают ASP.NET, разверните узел Компоненты разработки приложений и выберите ASP.NET 4.5 (если вы установили .NET 3.5, также выберите ASP.NET 3.5).
\\Автоматически будут выбраны следующие дополнительные компоненты:\\
	\begin{itemize}
	\item .NET Extensibility 4.5.
	\item Расширения ISAPI.
	\item Фильтры ISAPI.
	\item .NET Extensibility 3.5 (если была выбрана платформа ASP.NET 3.5).
	\end{itemize}
Нажмите кнопку ОК, чтобы закрыть диалоговое окно Компоненты Windows.
Чтобы убедиться, что службы IIS успешно установлены, введите в веб-браузере следующее:
\\\textbf{http://localhost}\\
Откроется страница приветствия IIS по умолчанию.
\end{enumerate}
\subsubsection{Установка служб IIS и модулей ASP.NET с помощью командной строки}
Введите следующую команду в командной строке или в скрипте:
\lstinputlisting[
frame=single,
breaklines=true,
inputencoding=cp1251]{sources/code.txt}

\subsection{Добавление приложения ASP.NET}
\subsubsection{Из пользовательского интерфейса}
\begin{enumerate}
\item Откройте Диспетчер IIS.
	\begin{itemize}
	\item При работе в Windows Server 2012 на начальной странице щелкните Диспетчер сервера, а затем нажмите кнопку ОК. В диспетчере сервера выберите меню Сервис, а затем выберите Диспетчер служб IIS.
	\item При работе в Windows 8 на начальной странице введите Панель управления, а затем в результатах поиска щелкните значок Панель управления. В окне Панель управления выберите Системы и безопасность, затем Администрирование, после чего выберите Диспетчер служб IIS.
	\end{itemize}
\item На панели Соединения разверните узел Сайты.
\item Правой кнопкой мыши щелкните сайт, для которого требуется создать приложение, и выберите Добавить приложение.
\item В поле Псевдоним введите значение для URL-адреса приложения, например marketing. Это значение используется в URL-адресе для доступа к приложению.
\item Щелкните Выбрать, если нужно выбрать пул приложений, отличный от пула, указанного в поле Пул приложений. В диалоговом окне Выбор пула приложений в списке Пул приложений выберите пул приложений, а затем нажмите кнопку ОК.
\item В поле Физический путь введите физический путь к папке приложения или нажмите кнопку обзора (...), чтобы перейти к файловой системе для поиска папки.
\item При необходимости щелкните Подключиться как, чтобы указать учетные данные, обладающие разрешением для доступа к физическому пути. Если не используются определенные учетные данные, выберите параметр Пользователь веб-приложения (сквозная проверка подлинности) в диалоговом окне Подключиться как.
\item Либо щелкните Проверка настройки, чтобы проверить все параметры, указанные для приложения.
\item Нажмите кнопку ОК.
\end{enumerate}
\subsubsection{Из командной строки}
Чтобы добавить приложение на сайт, используйте следующий синтаксис:\\
\textbf{appcmd add app /site.name:} строка \textbf{/path:} строка \textbf{/physicalPath:} строка
\\\\
Переменная \textbf{site.name} строка — это имя веб-сайта, на который нужно добавить приложение. Переменная \textbf{pathстрока} — это виртуальный путь к приложению, например \textbf{/application}, а \textbf{physicalPathстрока} — это физический путь к содержимому приложения в файловой системе.
\\\\
Например, чтобы добавить приложение \textbf{marketing} на сайт \textbf{contoso}, содержимое которого хранится в папке \textbf{c:\symbol{92} application}, в командной строке введите следующее, а затем нажмите клавишу Enter:\\\\
\textbf{appcmd add app /site.name: contoso /path:/ marketing /physicalPath:c:\ application}
\\\\\\
Дополнительные настройки параметров ASP.NET можно посмотреть в документации по ссылке:\\\\
\textbf{https://technet.microsoft.com/ru-ru/library/hh831626(v=ws.11).aspx}
\section{Установка под Debian и Ubuntu}
\subsection{Установка CoreCLR и/или Mono}
Для установки Mono согласно документации все что необходимо – это установить пакет mono-complete и сам run-time. Однако, кроме этого так же потребуется добавить источники пакетов для загрузки Mono и установить сертификаты сайтов, используемых для восстановления пакетов NuGet.

Проект с вспомогательными скриптами для установки .NET расположен на GitHub по адресу: https://github.com/VeselovAndrey/dotnet-install проект не является нашей разработкой, но распространяется под лицензией Apache License так что мы можем его использовать.

В дальнейшем предполагается что у пользователя Linux есть права записи с текущую директорию. Такой является его домашняя директория /home/<UserName>, в которую он обычно попадает после запуска терминала. Самостоятельно перейти в нее можно c помощью команды cd /home/<UserName>, где <UserName> это логин пользователя.

\textbf{На данный момент он содержит два скрипта:}
\begin{itemize}
\item debian-dotnet-install.sh – устанавливает CoreCLR, Mono, Node.js на Debian и Ubuntu
\item samba-share-dir.sh – создает директорию и открывает к ней доступ по сети
\end{itemize}

\textbf{Загрузить их можно 2 способами:}
\begin{itemize}
\item \textbf{1.Загрузка скрипта с помощью wget}\\
Этот способ подходит если необходимо загрузить только один скрипт. Например: debian-dotnet-install.sh. В терминале необходимо выполнить следующие 2 команды (они записаны в одной строке через \&\&):
\lstinputlisting[
firstline=1, 
lastline=1,
frame=single,
breaklines=true,
inputencoding=cp1251]{sources/code2.txt}
Утилита wget предназначена для загрузки файлов из сети. Поскольку в параметрах нет особых указаний, данные будут сохранены в текущую директорию. Если в ней уже существует файл с таким именем, то wget его перезапишет (ключ "-N"). После загрузки скрипту будут добавлены права на запуск (chmod +x).

\item \textbf{2.Загрузка скриптов с помощью git}\\
Поскольку проект размещён на GitHub его можно просто клонировать.
\lstinputlisting[
firstline=3, 
frame=single,
breaklines=true,
inputencoding=cp1251]{sources/code2.txt}
Здесь выполняются следующие действия:
\begin{itemize}
\item sudo позволяет запускать команды с привилегиями супер пользователя.
\item apt-get это менеджер пакетов Linux. Используя его, в систему добавляем git.
\item С помощью git клонируем проект в директорию "/home/<UserName>/dotnet-install".
\item Переходим в директорию проекта.
\item Разрешаем запуск всех скриптов вызовом chmod.
\end{itemize}

\textbf{Установка}\\\\
После того как загрузка завершена можно запустить установку .NET с помощью вызова debian-dotnet-install.sh. Скрипт всегда устанавливает CoreCLR, а так же ряд других компонентов, определяемых параметрами:
	\begin{itemize}
		\item $--$coreclr – устанавливает CoreCLR.
		\item $--$mono – устанавливает Mono.
		\item $--$nodejs – устанавливает Node.js.
		\item $--$help – выводит список доступных параметров.
	\end{itemize}
\end{itemize}
Пример вызова скрипта для установки CoreCLR, Mono и Node.js:\\\\
\textbf{user@server:~\$ ./debian-dotnet-install.sh $--$coreclr $--$mono $--$nodejs}\\\\
По завершению установки необходимо перезапустить терминал, выйдя из него командой exit. Так же, при установке Mono, потребуется подтверждение добавления сертификатов сайтов. К сожалению, ключ для автоматизации данного процесса у утилиты certmgr отсутствует.
\textbf{Дополнительно: настройка директории для обмена файлами}
Второй скрипт, samba-add-shared-dir.sh, помогает решить простую задачу – обмен файлами с компьютером разработчика. Например, это пригодиться для копирования проекта на тестовый сервер. Все что нужно – запустить скрипт с указанием имени директории. Она будет автоматически создана в каталоге /home/<UserName> и открыта для доступа по сети:\\\\
\textbf{user@server:~\$ ./samba-add-shared-dir.sh [имя директории] [сетевое имя]}\\\\
Второй параметр опционален и используется для задания сетевого имени, отличного от имени самой директории. Доступ осуществляется с авторизацией. При этом используется учетная запись пользователя запустившего скрипт. При этом скрипт запросит пароль для директории, который можно задать отличным от пароля для входа в систему (последний при этом изменен не будет).
\\\\Тестовый пример:\\\\
\textbf{user@server:~\$ git clone git://github.com/aspnet/home.git ~/aspnet-home\\
user@server:~\$ cd aspnet-home/samples/1.0.0.-rc1-update1/HelloMvc\\
user@server:~/aspnet-home/samples/1.0.0-rc1-update1/HelloMvc\$ dnu restore\\
user@server:~/aspnet-home/samples/1.0.0-rc1-update1/HelloMvc\$ dnx web}\\
Последняя команда запускает веб-сервер.
\subsection{Установка MuPDF}
Скачайте исходный код: \textbf{git clone http://mupdf.com/repos/mupdf.git}
Затем запустить в терминале: \textbf{make prefix=/usr/local install}
\end{document}
